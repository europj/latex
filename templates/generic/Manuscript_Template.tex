%===============================================================================
%
% LI GROUP GENERAL LATEX TEMPLATE
% Updated: Feb 20, 2014 by JW May
% Created: Jun 1, 2012 by JW May
%
%===============================================================================

%-------------------------------------------------------------------------------
% PREAMBLE AND DOCUMENT FORMATTING
%-------------------------------------------------------------------------------
\documentclass[12pt]{article}

% PACKAGES
\usepackage{amsmath}      % for equation typesetting
\usepackage{amssymb}      % for equation typesetting
\usepackage{wasysym}      % for geometric shapes
\usepackage{color}        % for colored fonts
\usepackage{setspace}     % for 1.5 and double spacing
\usepackage{overcite}     % for superscripted in-text citations
\usepackage{graphicx}     % main graphics package
\usepackage{wrapfig}      % allow text wrapping around figures
\usepackage[dvipsnames]{xcolor} % for inserting colored text
%\usepackage[user=XXX]{trkchg}  % provides commands for tracking changes in compiled document

% DOCUMENT FORMATTING
\usepackage[top=1in, bottom=1in, left=1in, right=1in]{geometry}   % set page margins
%\usepackage{indentfirst}               % uncomment to indent the first line of all paragraphs
%\pagestyle{empty}                      % uncomment to remove page numbers from entire document
%\onehalfspacing                        % uncomment for 1.5 spacing
%\doublespacing                         % uncomment for double spacing

% FONTS
\usepackage[T1]{fontenc}    % provides 8-bit encoding with fonts that have 256 glyphs
%\usepackage{times}         % uncomment to use Times New Roman font

% BIBLIOGRAPHY FORMATTING
\bibliographystyle{jpcnew}  % insert bibliography style file here

% CAPTION FORMATTING
\usepackage[font=footnotesize,labelfont=bf,labelsep=period,width=0.75\textwidth]{caption}   % format single-image captions and table titles
\usepackage[font=footnotesize,labelfont=bf,labelsep=period]{subcaption}                     % format subfigure captions
\DeclareCaptionSubType*[arabic]{figure}                   % use arabic numerals for subfigure captions (e.g., 1.1, 1.2, etc.)
\DeclareCaptionLabelFormat{subfiglabel}{Figure #2}        % append 'Figure' to subfigure captions (e.g., Figure 1.1, Figure 1.2, etc.)
\captionsetup[subfigure]{labelformat=subfiglabel,singlelinecheck=false}                     % format subfigure captions

% CROSS-REFERENCE FORMATTING
% For use with the cleveref package
% Define the format of Figure, Table, Equation, and Section cross-references in the text
\usepackage[capitalize]{cleveref}
\crefname{figure}{Fig.}{Figs.}
\Crefname{figure}{Figure}{Figures}
\crefname{table}{Tab.}{Tabs.}
\Crefname{table}{Table}{Tables}
\crefname{equation}{Eq.}{Eqs.}
\Crefname{equation}{Equation}{Equations}
\crefname{section}{Sec.}{Secs.}
\Crefname{section}{Section}{Sections}

% USER-DEFINED COMMANDS
%
% General / Formatting
\newcommand{\bd}[1]{\textbf{#1}} % bold text
\newcommand{\subsubsubsection}[1]{\vspace{5pt}\noindent{\underline{\emph{#1}}}\vspace{4pt}} % subsubsubsection command
%
% For quantum dot papers
\newcommand{\Mn}{Mn$^{2+}$}
\newcommand{\Co}{Co$^{2+}$}
\newcommand{\Zn}{Zn$^{2+}$}
\newcommand{\Cd}{Cd$^{2+}$}
\newcommand{\Se}{Se$^{2-}$}
\newcommand{\MLCT}{ML$_{\text{CB}}$CT}
\newcommand{\LMCT}{L$_{\text{VB}}$MCT}
\newcommand{\MnZnO}{Mn$^{2+}$:ZnO}
\newcommand{\CoZnO}{Co$^{2+}$:ZnO}
\newcommand{\MnCdSe}{Mn$^{2+}$:CdSe}
\newcommand{\CoCdSe}{Co$^{2+}$:CdSe}
\newcommand{\ecb}{$e_{\text{CB}}^{-}$}
\newcommand{\ecbp}{$e_\text{CB}^{-}$$^{\prime}$}
\newcommand{\hvb}{$h_{\text{VB}}^+$}
\newcommand{\sd}{$s$--$d$}
\newcommand{\pd}{$p$--$d$}
\newcommand{\spd}{$sp$--$d$}
\newcommand{\dqd}{d$_{\text{QD}}$}
%
% Chemistry shortcuts
\newcommand{\atom}[2]{#1$_{#2}$}                          % subscripted atom symbol
\newcommand{\term}[3]{$^{#1}$#2$_{#3}$}                   % term (level) symbol
%
% Mathematical Shortcuts
\newcommand{\pfrac}[2]{\frac{\partial #1}{\partial #2}}   % partial derivative
\newcommand{\difrac}[2]{\frac{d #1}{d #2}}                % derivative
\newcommand{\bpar}[1]{\left( #1 \right)}                  % big parentheses
\newcommand{\bbra}[1]{\left[ #1 \right]}                  % big brackets
\newcommand{\bbar}[1]{\left| #1 \right|}                  % big bars
\newcommand{\bra}[1]{\left\langle #1 \right\vert}         % bra
\newcommand{\ket}[1]{\left\vert #1 \right\rangle}         % ket
\newcommand{\inner}[2]{\left\langle #1 \left\vert\right. #2 \right\rangle}            % bracket
\newcommand{\innerop}[3]{\left\langle #1 \left\vert #2 \right\vert #3 \right\rangle}  % operator matrix element
\newcommand{\innersub}[4]{\langle \bd{#1}_{#2}, \bd{#3}_{#4} \rangle}                 % bracket with subscripts
\newcommand{\half}{\frac{1}{2}}                           % 1/2
\newcommand{\powfrac}[3]{\bpar{\frac{#1}{#2}}^{#3}}       % fraction raised to a power
\newcommand{\ii}{\infty}                                  % infinity symbol
\newcommand{\tquad}{\quad\quad\quad}                      % triple-quad spacing
\renewcommand{\Im}{\text{Im}}                             % imaginary symbol
\renewcommand{\Re}{\text{Re}}                             % real symbol
%-------------------------------------------------------------------------------
% Add your own commands below this line
%-------------------------------------------------------------------------------
\newcommand*\mycommand[1]{\texttt{\emph{#1}}}

%===============================================================================
%
% MAIN DOCUMENT BEGINS
%
%===============================================================================
\begin{document}

%-------------------------------------------------------------------------------
% TITLE
%-------------------------------------------------------------------------------
\title{Document Title}
\author{First Author, Second Author, Xiaosong Li$^*$ \\[12pt]
\emph{Department of Chemistry, University of Washington, Seattle, WA 98195} \\[12pt]
\texttt{email: li@chem.washington.edu}}
\date{}
\maketitle

%-------------------------------------------------------------------------------
% ABSTRACT
%-------------------------------------------------------------------------------
\begin{abstract}
	Enter abstract text here.
\end{abstract}
\newpage

%-------------------------------------------------------------------------------
% INTRODUCTION
%-------------------------------------------------------------------------------
\section*{Introduction}

Type introduction here.  Insert references as such.\cite{Li12_2898, Li12_22A512, Li12_1374, Li12_11223, Li11_144102, Li11_024118, Li06_835, GDVH21}

%-------------------------------------------------------------------------------
% METHODS
%-------------------------------------------------------------------------------
\section*{Methods}

Type methods here.

%-------------------------------------------------------------------------------
% RESULTS AND DISCUSSION
%-------------------------------------------------------------------------------
\section*{Results and Discussion}

Present results and type discussion here.

%-------------------------------------------------------------------------------
% CONCLUSION
%-------------------------------------------------------------------------------
\section*{Conclusion}

Type conclusion here. \\[12pt]

%-------------------------------------------------------------------------------
% ACKNOWLEDGEMENTS
%-------------------------------------------------------------------------------
\noindent{\Large\textbf{Acknowlegement}} \\[12pt]
This work was supported by AGENCY. Computations were facilitated through the use of advanced computational, storage, and networking infrastructure provided by the Hyak supercomputer system at the University of Washington, funded by the Student Technology Fee.

%-------------------------------------------------------------------------------
% REFERENCES
%-------------------------------------------------------------------------------
\newpage % new page
%\pagestyle{empty} % uncomment to remove page numbers from references page only
%\renewcommand{\refname}{Type your custom reference section header here} % uncomment to define header for references section; default is "References"
\bibliography{sample} % add .bib files to this list

%===============================================================================
%
% FLOATS: FIGURES AND TABLES
%
% *** REMOVE THIS SECTION BEFORE SUBMITTING PAPER ***
%
%===============================================================================
\newpage
\section*{Floats: Figures and Tables}

This section provides examples of inserting figures, subfigures, text-wrapped figures, and tables using the caption and subcaption packages. \\

\noindent\textbf{Single Figure Example}

%-------------------------------------------------------------------------------
% FIGURE
%-------------------------------------------------------------------------------
\begin{figure}[h]
	\begin{center}
		\includegraphics[width=3.25cm]{sample.pdf}
		\caption{Example of a single figure.  Control over the formatting of single figures may be done using the \texttt{caption} package.}
		\label{fig:singlefigure}
	\end{center}
\end{figure}

\noindent\textbf{Subfigure Example}

%-------------------------------------------------------------------------------
% SUBFIGURE
%-------------------------------------------------------------------------------
\begin{figure}[h]
	\begin{center}
	\captionsetup{labelformat=empty} % remove if you want the set of subcaptions to have one general caption
		\begin{subfigure}[t]{0.3\textwidth}
			\centering
			\includegraphics[width=\textwidth]{sample.pdf}
			\caption{First panel of subfigure.}
			\label{fig:firstsubfigure}
		\end{subfigure}
		\quad	% control spacing between first and second subfigure
		\begin{subfigure}[t]{0.3\textwidth}
			\centering
			\includegraphics[width=\textwidth]{sample.pdf}
			\caption{Second panel of subfigure.}
			\label{fig:secondsubfigure}
		\end{subfigure}	
		\quad	% control spacing between second and third subfigure
		\begin{subfigure}[t]{0.3\textwidth}
			\centering
			\includegraphics[width=\textwidth]{sample.pdf}
			\caption{Third panel of subfigure.}
			\label{fig:thirdsubfigure}
		\end{subfigure}
		\caption{} % leave blank with \captionsetup command above to remove the general figure caption leaving only subfigure captions
	\end{center}
\end{figure}

\noindent\textbf{Wrapped Figure Example} \\

Lorem ipsum dolor sit amet, consectetur adipiscing elit. Pellentesque congue nunc a augue mattis at pellentesque neque ultrices. Fusce dictum, sem nec semper vestibulum, sem odio malesuada libero, ut aliquet nibh mauris in sapien. Suspendisse congue erat at lorem placerat vel dictum nisi fringilla. Nam ligula lectus, ornare id mattis vitae, ornare quis arcu. Nulla facilisi. Duis commodo lorem lacinia odio tristique porttitor.

%-------------------------------------------------------------------------------
% WRAPPED FIGURE
%-------------------------------------------------------------------------------
% USAGE: \begin{wrapfigure}[1]{2}[3]{4}
%	1 = Number of lines (optional}
%	2 = "r" for right and "l" for left figure placement
%	3 = Overhang (optional)
%	4 = Width to be reserved
\begin{wrapfigure}{l}{3.5cm}
	\centering
	\includegraphics[width=3.25cm]{sample.pdf}
	\caption{Example of a wrapped figure.}
	\label{fig:wrappedfigure}	
\end{wrapfigure}

Quisque posuere libero ut nisl pellentesque in aliquam nunc venenatis. Praesent scelerisque mattis interdum. Ut massa enim, bibendum vitae semper at, accumsan vel magna. Pellentesque habitant morbi tristique senectus et netus et malesuada fames ac turpis egestas. Nunc blandit urna id lorem aliquet cursus. Nunc diam lectus, commodo sit amet consectetur eu, mollis id purus. Morbi dolor ante, vulputate eu ultricies non, consequat in sapien.

%===============================================================================
%	END OF EXAMPLE SECTION
%===============================================================================

\end{document}
